\documentclass{article}
\usepackage[utf8]{inputenc}
\usepackage{amsmath,amsfonts,amssymb,amsthm,epsfig,epstopdf,titling,url,array}
\usepackage{graphicx,amsmath, amssymb, amsthm, amsfonts}
\usepackage{IEEEtrantools}
\usepackage{cleveref}

\theoremstyle{plain}
\newtheorem{thm}{Theorem}[section]
\newtheorem{lem}[thm]{Lemma}
\newtheorem{prop}[thm]{Proposition}
\newtheorem*{cor}{Corollary}
\newtheorem*{prf}{proof}

\theoremstyle{definition}
\newtheorem{defn}{Definition}[section]
\newtheorem{conj}{Conjecture}[section]
\newtheorem{exmp}{Example}[section]

\theoremstyle{remark}
\newtheorem*{rem}{Remark}
\newtheorem*{note}{Note}
\title{$\S$8.1 Markov Inequality}
%\author{Lu Sun }
\date{April 03, 2019}

\usepackage{natbib}
\usepackage{graphicx}

\begin{document}

\maketitle

\section{Block Functions}

\begin{defn}
A analytic function $g(z)$ on $\mathbb{D}$ is called a \textbf{Block Function} if $$||g||_{B}=\sup_{z\in\mathbb{D}}|g'(z)|(1-|z|^2)<\infty$$
\end{defn}

\textbf{Remark}
\begin{itemize}
    \item $||g||_{B}$ is called Block norm.
    \item If $T(z)=\lambda\frac{z+a}{1+\Bar{a}z},a\in \mathbb{D},\lambda\in\partial\mathbb{D}$, then $||g\circ T||_B=||g||_B$.
    \item If $Re(g)$ is bounded, then $g\in B$, here $B$ is the set of Block Functions.
\end{itemize}

\begin{thm}
Let $g(z)$ be analytic on $\mathbb{D}$. Then $g\in B$ if and only if $g$ is Lipschitz continuous as a map from hyperbolic metric on $\mathbb{D}$ to the Euclidean metric on $\mathbb{D}$. Namely,
$$||g||_B=\sup_{z,w\in\mathbb{D}}\frac{|g(z)-g(w)|}{\rho(z,w)}$$
$$\rho_{\mathbb{D}}(z_1,z_2)=\inf\int^{z_2}_{z_1}\frac{|dz|}{1-|z|^2}$$
\end{thm}

\section{Law of Herated Logarithm}
In this section, law of Herated logarithm for Block functions is shown:
\begin{thm}\textbf{(Markov)} $\exists C>0$, $s.t.$ whenever $g\in B$ on $\mathbb{D}$ $a.e.$ on $\partial\mathbb{D}$,
$$\limsup_{r\to{1}}\frac{|g(re^{i\theta})|}{\sqrt{\log(\frac{1}{1-r})\log\log\log(\frac{1}{1-r})}}\leq C||g||_B$$
\end{thm}
\begin{thm}
If $||g||_B\leq1$ and if $\exists \beta >0$ and $M<\infty$, $s.t.$ for all $z_0\in\mathbb{D}$, 

\begin{IEEEeqnarray*}{rCl}
 \sup_{\{z:\rho(z,z_0)<M\}}(1-|z|^2)|g'(z)|
& \geq &\beta, \IEEEyessubnumber*
\end{IEEEeqnarray*}
then $a.e.$ on $\partial\mathbb{D}$
$$\limsup_{r\to{1}}\frac{Re\big(g(re^{i\theta})\big)}{\sqrt{\log(\frac{1}{1-r})\log\log\log(\frac{1}{1-r})}}\geq C(\beta,M)>0$$
\end{thm}
\begin{thm}
\textbf{Hardy-Littlewood maximal theorem}(H.L.theorem):

If $f\in L^p(R^n)$ for $n> 1, 1<p\leq\infty$, then $\exists$ a constant $C_{p,n}>0$, $s.t.$ $$\Bigg|\Bigg|\sup_{r>0}\Big|\frac{1}{|B(x,r)|}\int_{B(x,r)}f(y)dy\Big|\Bigg|\Bigg|_{L^p(R^n)}\leq C_{p,n}||f||_{L^p(R^n)}.$$ 
\end{thm}

\textbf{Remark}
\begin{itemize}
    \item If conformal map $\psi$ mapping $\mathbb{D}$ to the domain $\Omega$ inside the snowflake curve, then $(0a)$ holds for the global function $g=\log(\psi')$ and vice versa.
    \item If $g(z)=\sum_{n=1}^{\infty}z^{2^n}$, and $g_N(z)=\sum_{n=1}^{N}z^{2^n}$ is the partial sum of $g(z)$ with $\limsup_{r\to 1}\frac{g_N(re^{i\theta})}{\sqrt{N\log\log N}}=1$. Then,
    $$\limsup_{r\to{1}}\frac{|g(re^{i\theta})|}{\sqrt{\log(\frac{1}{1-r})\log\log\log(\frac{1}{1-r})}}=1$$
\end{itemize}
\begin{prf} Proof for Theorem 2.1.

WLOG: $g(0)=0$ and $||g||_B=1$.

Let $p\in\mathbb{N}$ and consider 
$$I_p(r)=\frac{1}{2\pi}\int^{2\pi}_{0}|g(re^{i\theta})|^{2p}d\theta,$$
then,
$$\frac{d}{dr}\big(rI_p'(r)\big)=\frac{4p^2r}{2\pi}\int^{2\pi}_{0}|g(re^{i\theta})|^{2p-2}*|g'(re^{i\theta})|^2d\theta.$$

By Hardy's inequality:
\begin{equation}
    I_p(r)\leq p!\big(\log(\frac{1}{1-r^2})\big)^p\leq p!\big(\log(\frac{1}{1-r^2})\big)^p,
\end{equation}
we have:
\begin{equation*}
  \begin{array}{lcl}
       \frac{d}{dr}\big(rI_p'(r)\big)&=&\frac{4p^2r}{2\pi}\int^{2\pi}_{0}|g(re^{i\theta})|^{2p-2}*|g'(re^{i\theta})|^2d\theta  \\
       &\overset{By\ ||g||_B=1}{\leq}& \frac{4p^2r}{(1-r)^2}I_{p-1}(r)\\
                     &\overset{By\ (1)}{\leq}& \frac{4pp!r}{(1-r^2)^2}\big(\log\frac{1}{1-r^2})^{p-1}\big)\\
              &\leq& p!\frac{d}{dr}\big(r\frac{d}{dr}(\log\frac{1}{1-r^2})^p\big)\\
  \end{array}  
\end{equation*}
by ind assumption integrating both sides two times yields $(1)$.

Applying H.L.theorem to $|g(re^{i\theta})|^p\in L^2$ with $g^*_r(e^{i\theta})=\sup_{\rho<r}|g(\rho e^{i\theta})|$, we get:
\begin{equation}
    \frac{1}{2\pi}\int^{2\pi}_{0}|g_r^*(e^{i\theta})|^{2p}d\theta\leq Cp!\big(\log\frac{1}{1-r}\big)^p.
\end{equation}

Let $\alpha>1$ and $A_p(r)=\frac{1}{1-r}\frac{1}{\big(\log\frac{1}{1-r}\big)^{p+1}}\frac{1}{\big(\log\log\frac{1}{1-r}\big)^{\alpha}}$, then we get:
\begin{equation}
    \int^{1}_{r}A_p(s)ds\geq\frac{C}{p}\frac{1}{\big(\log\frac{1}{1-r}\big)^{p+1}}\frac{1}{\big(\log\log\frac{1}{1-r}\big)^{\alpha}}.
\end{equation}

Consider:
\begin{equation}
    \begin{array}{lcl}
         \int^{1}_rA_p(s)
         \int^{2\pi}_{0}|g_r^*(e^{i\theta})|^{2p}d\theta
         ds
         &\overset{By(2)\ \&\ def\ of\ A_p(z)}{\leq}&
         Cp!\int^1_r\frac{1}{\big(\log\log\frac{1}{1-s}\big)^{\alpha}}\frac{1}{\log\frac{1}{1-s}}\frac{ds}{1-s}\\
         &\leq&C_{\alpha}p! 
    \end{array}
\end{equation}

Let $E_p:=\{\theta:\int^1_rA_p(s)|g_s^*(e^{i\theta})|^{2p}ds>C_{\alpha}p^2p!\}$. By Chebyshev and Fubini, we have $|E_p|\leq \frac{1}{p^2}$. If $\theta\notin \cup_{n>p}E_n$, then
\begin{equation}
    \begin{array}{lcl}
        |g(re^{i\theta})|^{2p} &\leq&\frac{\int^1_rA_p(s)|g^*_s(e^{i\theta})|^{2p}ds}{\int^1_rA_p(s)ds}  \\
         &\overset{By\ (3)\ and\ E_p}{\leq}&C_{\alpha}p^2p!\frac{p(\log\frac{1}{1-r})^p(\log\log\frac{1}{1-r})^{\alpha}}{C} 
    \end{array}
\end{equation}

By (5) we have:
\begin{equation}
    \frac{|g(re^{i\theta})|}{\sqrt{\log(\frac{1}{1-r})\log\log\log(\frac{1}{1-r})}}
    \leq
    \frac{C^{-\frac{1}{2p}}C_{\alpha}^{-\frac{1}{2p}}p^{\frac{3}{2p}}(p!)^{\frac{1}{2p}}(\log\log\frac{1}{1-r})^{\frac{\alpha}{2p}}}{\sqrt{\log\log\log\frac{1}{1-r}}}=(*)
\end{equation}

By stirling's formula:
$$p!\sim \sqrt{2\pi p}\big(\frac{p}{e}\big)^p,$$
and setting $p=\log\log\log\frac{1}{1-r}$, we have

\begin{equation}
\begin{array}{lcl}
    (*) & =& \frac{
          C^{-\frac{1}{2p}}
          C_{\alpha}^{-\frac{1}{2p}}
          p^{\frac{3}{2p}}
        \big(
          \sqrt{2\pi p}
          (\frac{p}{e})^p
        \big)^{\frac{1}{2p}}
          (e^p)^{\frac{\alpha}{2p}}
          }
          {\sqrt{p}} \\
     &=& \big(\frac{\sqrt{2\pi}}{CC_{\alpha}}\big)^{\frac{1}{2\alpha}}
           p^{\frac{3}{2p}}(\sqrt{e})^{\alpha-1}\sim(\sqrt{e})^{\alpha-1}
\end{array}
           \end{equation}
With $\alpha>1$, $||g||_B=1$, we get the constant with $C=1,|E_p|<\frac{1}{2p}\to 0.$ Finally, by equation $(6)\ (7)$, we get the Markov inequality in Theorem 2.1.

\end{prf}

\end{document}
